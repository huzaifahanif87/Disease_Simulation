\documentclass{beamer}
\usetheme{Madrid}
\usecolortheme{spruce}

\usepackage{listings}
\usepackage{graphicx}
\usepackage{hyperref}

\definecolor{codegreen}{rgb}{0,0.6,0}
\definecolor{codegray}{rgb}{0.5,0.5,0.5}
\definecolor{codepurple}{rgb}{0.58,0,0.82}
\definecolor{backcolour}{rgb}{0.95,0.95,0.92}

\lstdefinestyle{mystyle}{
    backgroundcolor=\color{backcolour},   
    commentstyle=\color{codegreen},
    keywordstyle=\color{magenta},
    numberstyle=\tiny\color{codegray},
    stringstyle=\color{codepurple},
    basicstyle=\ttfamily\footnotesize,
    breakatwhitespace=false,         
    breaklines=true,                 
    captionpos=b,                    
    keepspaces=true,                 
    showspaces=false,                
    showstringspaces=false,
    showtabs=false,                  
    tabsize=2
}

\lstset{style=mystyle}

\title{Infectious Disease Spread Simulation}
\subtitle{KD-Tree Optimized Graph-Based Modeling}
\author{Your Name Here}
\institute{Design and Analysis of Algorithms}
\date{\today}

\begin{document}

\begin{frame}
\titlepage
\end{frame}

\begin{frame}{Outline}
\tableofcontents
\end{frame}

\section{Introduction}
\begin{frame}{Introduction}
\begin{itemize}
    \item Real-time simulation of disease spread through social interactions
    \item Graph-based approach modeling human connections and proximity
    \item Optimized using KD-trees for efficient spatial queries
    \item Goal: Create realistic model while maintaining computational efficiency
\end{itemize}

\vspace{0.5cm}
\begin{center}
    \framebox{\includegraphics[width=0.7\textwidth]{screenshot1.png}}\\
    \small{Simulation showing healthy (green), infected (red),\\ quarantined (purple), and recovered (blue) individuals}
\end{center}
\end{frame}

\section{System Architecture}
\begin{frame}{System Architecture}
\textbf{Components:}
\begin{itemize}
    \item \textbf{Graph Model}: Social network with family, social, and random connections
    \item \textbf{Spatial Engine}: KD-tree optimized position tracking 
    \item \textbf{Disease Model}: State-based infection logic
    \item \textbf{Visualization}: Real-time animation and statistics
\end{itemize}

\vspace{0.3cm}
\textbf{Disease States:}
\begin{itemize}
    \item Healthy → Infected → Detected → Recovered
    \item Each state transition follows probabilistic rules
\end{itemize}

\vspace{0.3cm}
\begin{center}
    \framebox{\includegraphics[width=0.6\textwidth]{architecture.png}}\\
    \small{Component interaction diagram}
\end{center}
\end{frame}

\section{Algorithmic Optimizations}
\begin{frame}{Algorithmic Optimizations}
\textbf{Challenge: Performance Bottleneck}
\begin{itemize}
    \item Initial proximity checking: O(n²) complexity
    \item Quadratic scaling made large simulations impractical
\end{itemize}

\vspace{0.3cm}
\textbf{Solution: KD-Tree Spatial Partitioning}
\begin{itemize}
    \item Reduced proximity queries from O(n²) to O(n log n)
    \item Enables efficient radius-based neighbor finding
\end{itemize}
\end{frame}

\begin{frame}[fragile]{KD-Tree Implementation in Movement}
\begin{lstlisting}[language=Python]
# Key optimization in movement.py
def update_positions(G, positions, velocities, recovered):
    # Create KD-Tree for efficient spatial queries
    node_list = list(G.nodes())
    pos_array = np.array([positions[n] for n in node_list])
    tree = KDTree(pos_array)
    
    # Find nearby nodes in O(log n) time instead of O(n)
    nearby_idxs = tree.query_ball_point(pos, repulsion_radius)
\end{lstlisting}
\end{frame}

\begin{frame}[fragile]{KD-Tree Implementation in Infection Logic}
\begin{lstlisting}[language=Python]
# Key optimization in infection.py
def update_infection(G, positions, infected, ...):
    # Build KDTree for fast radius queries
    pos_array = np.array([positions[n] for n in all_nodes])
    tree = KDTree(pos_array)

    # Query all neighbors within infection radius - O(log n)
    neighbors = tree.query_ball_point(pos, INFECTION_RADIUS)
\end{lstlisting}
\end{frame}

\section{Model Implementation}
\begin{frame}{Model Implementation}
\textbf{Social Structure:}
\begin{itemize}
    \item Family connections (high infection probability: 15\%)
    \item Social connections (medium probability: 5\%)
    \item Proximity-based transmission within 10 units
\end{itemize}

\vspace{0.3cm}
\textbf{Infection \& Recovery Logic:}
\begin{itemize}
    \item Time-based recovery (23 simulated hours)
    \item Delayed detection (4-6 days)
    \item Quarantine option for detected cases
\end{itemize}

\vspace{0.3cm}
\begin{center}
    \framebox{\includegraphics[width=0.6\textwidth]{network.png}}\\
    \small{Visualization showing infection spread network patterns}
\end{center}
\end{frame}

\section{Results \& Conclusions}
\begin{frame}{Results \& Conclusions}
\textbf{Experimental Observations:}
\begin{itemize}
    \item Baseline (No intervention): Rapid spread through \textasciitilde80\% of population
    \item With quarantine: Moderate reduction in infections
    \item With social distancing (increased repulsion): Significant reduction
\end{itemize}

\vspace{0.3cm}
\textbf{Key Findings:}
\begin{itemize}
    \item KD-tree optimization reduced complexity from O(n²) to O(n log n)
    \item Enabled simulation of larger populations (300+ individuals)
    \item Social distancing more effective than reactive quarantine
\end{itemize}

\begin{center}
    \framebox{\includegraphics[width=0.7\textwidth]{results.png}}\\
    \small{Comparison graph showing infection curves under different scenarios}
\end{center}
\end{frame}

\begin{frame}{Thank You!}
\begin{center}
    \Large{Thank you for your attention!}
    
    \vspace{0.8cm}
    \framebox{\includegraphics[width=0.8\textwidth]{final.png}}\\
    \small{Final simulation state showing disease progression}

    \vspace{0.8cm}
    \normalsize{Questions?}
\end{center}
\end{frame}

\end{document}
